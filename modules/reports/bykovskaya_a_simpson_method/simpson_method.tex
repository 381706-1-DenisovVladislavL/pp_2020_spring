\documentclass{report}

\usepackage[T2A]{fontenc}
\usepackage[utf8]{luainputenc}
\usepackage[english, russian]{babel}
\usepackage[pdftex]{hyperref}
\usepackage[14pt]{extsizes}
\usepackage{listings}
\usepackage{color}
\usepackage{geometry}
\usepackage{enumitem}
\usepackage{multirow}
\usepackage{graphicx}
\usepackage{indentfirst}
\usepackage{caption}

\geometry{a4paper,top=2cm,bottom=3cm,left=2cm,right=1.5cm}
\setlength{\parskip}{0.5cm}
\setlist{nolistsep, itemsep=0.3cm,parsep=0pt}

\lstset{language=C++,
		basicstyle=\footnotesize,
		keywordstyle=\color{blue}\ttfamily,
		stringstyle=\color{red}\ttfamily,
		commentstyle=\color{green}\ttfamily,
		morecomment=[l][\color{magenta}]{\#}, 
		tabsize=4,
		breaklines=true,
  		breakatwhitespace=true,
  		title=\lstname,       
}

\makeatletter
\renewcommand\@biblabel[1]{#1.\hfil}
\makeatother

\begin{document}

\begin{titlepage}

\begin{center}
Министерство науки и высшего образования Российской Федерации
\end{center}

\begin{center}
Федеральное государственное автономное образовательное учреждение высшего образования \\
Национальный исследовательский Нижегородский государственный университет им. Н.И. Лобачевского
\end{center}

\begin{center}
Институт информационных технологий, математики и механики
\end{center}

\vspace{4em}

\begin{center}
\textbf{\LargeОтчет по лабораторной работе} \\
\end{center}
\begin{center}
\textbf{\Large«Вычисление многомерных интегралов методом Симпсона»} \\
\end{center}

\vspace{4em}

\newbox{\lbox}
\savebox{\lbox}{\hbox{text}}
\newlength{\maxl}
\setlength{\maxl}{\wd\lbox}
\hfill\parbox{7cm}{
\hspace*{5cm}\hspace*{-5cm}\textbf{Выполнил:} \\ студент группы 381708-1 \\ Быковская А. П.\\
\\
\hspace*{5cm}\hspace*{-5cm}\textbf{Проверил:}\\ доцент кафедры МОСТ, \\ кандидат технических наук \\ Сысоев А. В.
}

\vspace{\fill}

\begin{center} Нижний Новгород \\ 2020 \end{center}

\end{titlepage}

\setcounter{page}{2}

\tableofcontents
\newpage

\section*{Введение}
\addcontentsline{toc}{section}{Введение}
Метод Симпсона относится к приемам численного интегрирования. Суть метода заключается в приближении подынтегральной функции на отрезке $[a,b]$ интерполяционным многочленом второй степени $p_2(x)$.
\par Целью данной работы является изучение применения метода Симпсона в вычислении многомерных интегралов с использованием различных технологий для выполнения параллельных вычислений.
\newpage

\section*{Постановка задачи}
\addcontentsline{toc}{section}{Постановка задачи}
В рамках данных лабораторных работ ставится задача реализации метода Симпсона для вычисления многомерных интегралов с использованием различных технологий распараллеливания вычислений.
\par Для решения этой задачи требуется:
\begin{itemize}
\item Изучить и реализовать последовательный алгоритм метода Симпсона
\item Реализовать параллельный алгоритм, используя OpenMP.
\item Реализовать параллельный алгоритм, используя TBB.
\item Реализовать параллельный алгоритм, используя std::thread.
\item Проверить корректность работы и сравнить эффективность реализаций.
\end{itemize}

\newpage

\section*{Метод решения}
\addcontentsline{toc}{section}{Метод решения}
Суть метода Симпсона для вычисления интегралов заключается в следующем:
\par Отрезок $[a,b]$ разбивается на $2N$ равных частей, вычисляются значения функции $f_i=f(a + \frac{(b-a)i}{2N})$.
\par Полученные значения функции подставляем в формулу $\int_{a}^{b}{f(x)dx} \approx \frac{h}{3}(f(x_{0}) + 2\sum_{i=1}^{N-1}{f(x_{2i})} + 4\sum_{i=1}^{N}{f(x_{2i-1}} + f(x_{2N}))$, где $h=\frac{b-a}{2N}$ - величина шага.
\indent D~--- область интегрирования размерности $n$.

\newpage

\section*{Схема распараллеливания}
\addcontentsline{toc}{section}{Схема распараллеливания}
Общая схема работы параллельного алгоритма:
\begin{enumerate}
\item $2N$ точек разбиваются на $k$ частей, где $k$ - это количество потоков, на которые мы хотим распараллелить вычисления.
\item В каждом потоке вычисляются значения функций в точках из текущей части.
\item В мастер потоке суммируются значения, полученные на каждом отдельно взятом потоке.
\end{enumerate}
\newpage

\section*{Описание программной реализации}
\addcontentsline{toc}{section}{Описание программной реализации}
\subsection*{OpenMP}
\addcontentsline{toc}{subsection}{OpenMP}
Код, который должен выполняться параллельно, расположен внутри директивы  \verb|#pragma omp| \verb|parallel|.
\par С помощью директивы \verb|for|, которая позволяет распараллеливать циклы, вычисляются значения функций. Разбиение итераций между потоками происходит в зависимости от параметра \verb|schedule| (по-умолчанию static).

\subsection*{TBB}
\addcontentsline{toc}{subsection}{TBB}
С помощью функции \verb|tbb::parallel_for()|, которая позволяет распараллеливать циклы, вычисляются значения функций. Разбиение итераций между потоками происходит в зависимости от параметра \verb|tbb::blocked_range|. Разделение на порции вычислений происходит автоматически благодаря планировщику TBB.

\subsection*{std::thread}
\addcontentsline{toc}{subsection}{std::thread}
Для начала необходимо вручную создать объект потока \verb|std::thread| и вычислить объем порции для каждого потока: $\frac{2N}{k}$, где $k$ - количество потоков, может быть определено с помощью функции \verb|std::thread::hardware_concurrency|.
\par Затем создаются потоки, конструктор  принимает указатель на функцию, которую будет исполнять, и параметры это функции.
\par Для ожидания завершения вычислений каждого потока мастер-потоком для каждого потока вызывается метод \verb|join()|. 
\newpage

\section*{Подтверждение корректности}
\addcontentsline{toc}{section}{Подтверждение корректности}
Для подтверждения корректности в программе реализован набор тестов, разработанных при помощи библиотеки для модульного тестирования Google C++ Testing Framework. Проверяются случаи вычисления одномерных, двумерных и трехмерных интегралов. Значения, полученные при вычислении методом Симпсона, сравниваются со значениями, полученным аналитически.
\par Успешное прохождение всех тестов является подтвержением корректной работы программы. 

\newpage

\section*{Результаты экспериментов}
\addcontentsline{toc}{section}{Результаты экспериментов}
Конфигурация системы:
\begin{itemize}
\item Процессор: Intel(R) Core(TM) i5-6300H CPU @ 2.40GHz
\item Число ядер: 2
\item ОС: Windows 10
\end{itemize}

\par Значение функций вычисляется на 400 точках. 
\par Результаты экспериментов представлены в Таблице 1.

\begin{table}[htbp]
        \centering
        \begin{tabular}{|c|c|}
            \hline
                             & 2 \\ \hline
            Последовательная & 8.03 \\ \hline
            OpenMP           & 4.21 \\ \hline
            TBB              & 4.23 \\ \hline
            std::thread      & 4.01 \\ \hline
        \end{tabular}
        \caption{Результаты экспериментов.}
        \label{tab:results}
        По вертикали - число потоков. \\
        По горизонтали - тестируемая реализация. \\
        Время указано в секундах.
    \end{table}

\par По данным, полученным в результате экспериментов, можно сделать вывод, что параллельный алгоритм работает быстрее, чем последовательный приблизительно в n раз, где n ~--- количество потоков, при условии, что количество потоков меньше или равно количеству ядер процессора.
\newpage

\section*{Заключение}
\addcontentsline{toc}{section}{Заключение}
В результате выполнения данных лабораторных работ был изучен метод Симпсона для вычисление многомерных интегралов.
Реализованы последовательная и параллельные версии алгоритма с использованием технологий: OpenMP, TBB, std::thread.
\par Разработаны тесты подтверждающие корректность работы программы.
\par В заключение проведены эксперименты, подтверждающие эффективность параллельных версий алгоритма. По результатам опытов лучшей технологией для реализации параллельного метода Симпсона для вычисление многомерных интегралов оказалась std::thread.
\newpage

\begin{thebibliography}{1}
\addcontentsline{toc}{section}{Список литературы}
\bibitem{Sysoev} Сысоев А.В., Мееров И.Б., Свистунов А.Н., Курылев А.Л., Сенин А.В., Шишков А.В., Корняков К.В., Сиднев А.А. «Параллельное программирование в системах с общей памятью. Инструментальная поддержка». Учебно-методические материалы по программе повышения квалификации «Технологии высокопроизводительных вычислений для обеспечения учебного процесса и научных исследований». Нижний Новгород, 2007, 110 с. 
\bibitem{Barkalov} Баркалов К.А. Методы параллельных вычислений. Н. Новгород: Изд-во Нижегородского госуниверситета им. Н.И. Лобачевского, 2011
\bibitem{Wiki1} Wikipedia: the free encyclopedia [Электронный ресурс] // URL: https://en.wikipedia.org/wiki/\verb|Mersenne_Twister| (дата обращения: 20.03.2020)
\end{thebibliography}
\newpage

\section*{Приложение}
\addcontentsline{toc}{section}{Приложение}
\centerline{\bfseries Исходный код.} 


\lstinputlisting[language=C++, caption=Последовательная версия. Заголовочный файл]{../../../../modules/task_1/bykovskaya_a_simpson_method/simpson_method.h}
\lstinputlisting[language=C++, caption=Последовательная версия. Cpp файл]{../../../../modules/task_1/bykovskaya_a_simpson_method/simpson_method.cpp}
\lstinputlisting[language=C++, caption=Последовательная версия. Тесты]{../../../../modules/task_1/bykovskaya_a_simpson_method/main.cpp}

\lstinputlisting[language=C++, caption=OpenMP версия. Заголовочный файл]{../../../../modules/task_2/bykovskaya_a_simpson_method/simpson_method.h}
\lstinputlisting[language=C++, caption=OpenMP версия. Cpp файл]{../../../../modules/task_2/bykovskaya_a_simpson_method/simpson_method.cpp}
\lstinputlisting[language=C++, caption=OpenMP версия. Тесты]{../../../../modules/task_2/bykovskaya_a_simpson_method/main.cpp}

\lstinputlisting[language=C++, caption=TBB версия. Заголовочный файл]{../../../../modules/task_3/bykovskaya_a_simpson_method/simpson_method.h}
\lstinputlisting[language=C++, caption=TBB версия. Cpp файл]{../../../../modules/task_3/bykovskaya_a_simpson_method/simpson_method.cpp}
\lstinputlisting[language=C++, caption=TBB версия. Тесты]{../../../../modules/task_3/bykovskaya_a_simpson_method/main.cpp}

\lstinputlisting[language=C++, caption=std::thread версия. Заголовочный файл]{../../../../modules/task_4/bykovskaya_a_simpson_method/simpson_method.h}
\lstinputlisting[language=C++, caption=std::thread версия. Cpp файл]{../../../../modules/task_4/bykovskaya_a_simpson_method/simpson_method.cpp}
\lstinputlisting[language=C++, caption=std::thread версия. Тесты]{../../../../modules/task_4/bykovskaya_a_simpson_method/main.cpp}


\end{document}
