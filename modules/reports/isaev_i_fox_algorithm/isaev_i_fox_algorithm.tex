\documentclass{report}

\usepackage[T2A]{fontenc}
\usepackage[utf8]{luainputenc}
\usepackage[english, russian]{babel}
\usepackage[pdftex]{hyperref}
\usepackage[14pt]{extsizes}
\usepackage{listings}
\usepackage{amsmath}
\usepackage{color}
\usepackage{multicol}
\usepackage{longtable}
\usepackage{geometry}
\usepackage{enumitem}
\usepackage{multirow}
\usepackage{graphicx}
\usepackage{indentfirst}
\usepackage{caption}

\geometry{a4paper,top=2cm,bottom=2cm,left=2.5cm,right=1.5cm}
\setlength{\parskip}{0.5cm}

\lstset{language=C++,
		basicstyle=\footnotesize,
		keywordstyle=\color{blue}\ttfamily,
		stringstyle=\color{red}\ttfamily,
		commentstyle=\color{green}\ttfamily,
		morecomment=[l][\color{magenta}]{\#}, 
		tabsize=4,
		breaklines=true,
  		breakatwhitespace=true,
  		title=\lstname,       
}
\renewcommand{\thesubsection}{\arabic{subsection}}
\makeatletter
\def\@seccntformat#1{\@ifundefined{#1@cntformat}%
   {\csname the#1\endcsname\quad}
   {\csname #1@cntformat\endcsname}}
\newcommand\section@cntformat{}
\makeatother

\begin{document}

\begin{titlepage}

\begin{center}
МИНИСТЕРСТВО ОБРАЗОВАНИЯ И НАУКИ РОССИЙСКОЙ ФЕДЕРАЦИИ \\
Федеральное государственное автономное образовательное учреждение высшего образования\\
\textbf{«Нижегородский государственный университет им. Н.И. Лобачевского»} \\
\textbf{Национальный исследовательский университет} \\
\textbf{Институт информационных технологий математики и механики}\\
\end{center}

\vskip 3cm

\begin{center}
\textbf{\LargeОТЧЕТ}
\end{center}

\begin{center}
\textbf{\Large«Умножение плотных матриц. Элементы типа double. Блочная схема, алгоритм Фокса»}
\end{center}

\vskip 1cm

\newbox{\lbox}
\savebox{\lbox}{\hbox{text}}
\newlength{\maxl}
\setlength{\maxl}{\wd\lbox}
\hfill\parbox{7cm}{
\hspace*{5cm}\hspace*{-5cm}\textbf{Выполнил:} \\ студент группы 381706-2 \\ Исаев И.А. \\
\\
\hspace*{5cm}\hspace*{-5cm}\textbf{Проверил:}\\ доцент кафедры МОСТ, \\ кандидат технических наук \\ Сысоев А. В.}

\vskip 7cm

\begin{center}
Нижний Новгород \\
2020
\end{center}

\end{titlepage}

\tableofcontents
\setcounter{page}{2}

\newpage

\section*{Введение}
\addcontentsline{toc}{section}{Введение}
Умножение матриц – это одна из основных математических операций. Существует множество различных алгоритмов умножения матриц. Сложность вычисления произведения матриц по определению составляет  \( O \left( n^{3} \right)  \) , что неэффективно. Ускорить вычисления возможно, если применить параллельные алгоритмы умножения матриц.\par

Одним из таких параллельных алгоритмов является алгоритм Фокса. Стоит отметить, что данных алгоритм предназначен для распараллеливания для систем с распределённой памятью, однако это вполне применимо и для систем с общей памятью.\par

В рамках данной лабораторной работы мы реализуем алгоритм Фокса, используя различные технологии распараллеливания.\par

\newpage

\section*{Постановка задачи}
\addcontentsline{toc}{section}{Постановка задачи}
В рамках данной лабораторной работы нам необходимо разработать следующие компоненты:\par

\begin{itemize}
	\item Реализация последовательного алгоритма умножения матриц\par

	\item Реализация алгоритма Фокса с использованием OpenMP\par

	\item Реализация алгоритма Фокса с использованием TBB\par

	\item Реализация алгоритма Фокса с использованием Threads C++11\par

	\item Разработка модульных тестов для подтверждения корректности данной программной реализации
\end{itemize}\par

\newpage

\section*{Описание алгоритма}
\addcontentsline{toc}{section}{Описание алгоритма}
Алгоритм Фокса основан на блочном разбиении матрицы между вычислителями (checkboard distribution).\par

Пусть  \( A= \left( a_{ij} \right) ,~~ B= \left( b_{ij} \right) ,~ i,j=\overline{0, n-1} \)  – квадратные матрицы размера  \( n \times n \) . Тогда:\par

 \[ C=A \times B,~ c_{ij}=a_{i0}b_{0j}+a_{i1}b_{1j}+ \ldots +a_{i,n-1}b_{n-1,j}= \sum _{k=0}^{n-1}a_{ik}b_{kj} \] \par

 \( c_{ij} \)  – результат скалярного произведения i-ой строки матрицы A и j-ого столбца матрицы B.\par

 Пусть число вычислителей равно 4, а n = 4, тогда блочное разбиение будет выглядеть следующим образом:\par

 \[ A_{00}= \left( \begin{matrix}
a_{00}  &  a_{01}\\
a_{10}  &  a_{11}\\
\end{matrix}
 \right) ,~~A_{01}= \left( \begin{matrix}
a_{02}  &   a_{03}\\
a_{12}  &   a_{13}\\
\end{matrix}
 \right)  \]  \[ A_{10}= \left( \begin{matrix}
a_{20}  &  a_{21}\\
a_{30}  &  a_{31}\\
\end{matrix}
 \right) ,~~A_{11}= \left( \begin{matrix}
a_{22}  &  a_{23}\\
a_{32}  &  a_{33}\\
\end{matrix}
 \right)  \] \par


\vspace{\baselineskip}
 Аналогично для B и C.\par

Сам алгоритм Фокса заключается в том, что на каждом шаге нужно выбрать подматрицу A в каждой строке, чтобы каждый вычислитель умножил полученную матрицу на имеющуюся подматрицу B, а затем отправил свою подматрицу B вышестоящему в схеме вычислителю.\par

Так как мы реализуем данный алгоритм на потоках, то нужды в передаче-получении подматриц у нас нет, что упрощает алгоритм.\par

\newpage

\section*{Схема распараллеливания}
\addcontentsline{toc}{section}{Схема распараллеливания}
\subsection{OpenMP}
Каждый поток в OMP- реализации каждому потоку присваивается некоторая характеристика – координаты в схеме. Они показывают, в какой блок результирующей матрицы будет записан ответ по завершении работы потока. На каждой итерации step от 0 до q, где  \( q=\sqrt[]{num \_ threads} \) , нужный индекс высчитывается по формуле k\_bar = (thread\_i+step) $\%$  q. Перемножаем соответствующие блоки в строке и столбце и суммируем с результирующей матрицей. Для распараллеливания используется директива $\#$ pragma omp parallel.\par

\subsection{TBB}
Реализация в TBB, по большому счёту, схожа с таковой на OpenMP, разве что она основана на структуре blocked\_range2d и функции parallel\_for, которая принимает lambda-функцию, работающую точно так же, как и блок в OMP версии.\par

\subsection{Threads C++11}
В данной реализации мы создаём некоторое число потоков, которым при конструировании в качестве параметра передаём lambda-функцию (логика её работы не изменилась).\par

\newpage

\section*{Проверка корректности}
\addcontentsline{toc}{section}{Проверка корректности}
Для проверки корректности была использована библиотека модульного тестирования Google C++ Testing Framework, которая значительно упрощает написание тестов для программы.\par

\newpage

\section*{Результаты экспериментов}
\addcontentsline{toc}{section}{Результаты экспериментов}
Тестовая система:\par

\begin{itemize}
	\item Операционная система: Ubuntu 20.04 LTS (Focal Fossa)\par

	\item Процессор: Intel Core i5-3210M\par

	\item RAM: 6 Gb
\end{itemize}\par

Получили следующий результат:\par

\begin{table}[!h]
\resizebox{\textwidth}{!}{
\begin{tabular}{|c|c|c|c|c|c|c|c|}
\hline
\multirow{2}{*}{Размер матрицы} & Последовательно & \multicolumn{2}{c|}{OpenMP} & \multicolumn{2}{c|}{TBB} & \multicolumn{2}{c|}{Threads C++11} \\ \cline{2-8} 
     & Время, сек & Время, сек & Ускорение & Время, сек & Ускорение & Время, сек & Ускорение \\ \hline
500  & 2.64       & 1.23       & 2.14      & 1.24       & 2.12      & 1.22       & 2.16      \\ \hline
1000 & 21.92      & 12.41      & 1.76      & 12.24      & 1.79      & 11.75      & 1.86      \\ \hline
1500 & 70.21      & 43.17      & 1.62      & 44.50      & 1.57      & 43.68      & 1.60      \\ \hline
\end{tabular}
}
\caption{Результаты экспериментов}
\end{table}

\vspace{\baselineskip}
Как мы можем видеть, результаты с применением разных технологий примерно идентичны, и отклонения можно списать на погрешность. Связано это с тем, что способ распараллеливания, предложенный нами, будет выглядеть примерно одинаково с точки зрения логики в каждой из версий.\par

\newpage

\section*{Заключение}
\addcontentsline{toc}{section}{Заключение}
В результате данной лабораторной работы был изучен алгоритм Фокса параллельного умножения матриц, получен навык работы с OpenMP, TBB, Threads C++11. Поставленные задачи были решены, был проведён эксперимент, сравнивающий время выполнения при использовании разных технологий распараллеливания.\par

\newpage

\begin{thebibliography}{}
\addcontentsline{toc}{section}{Список литературы и ссылок}
\bibitem{Sysoev} Сысоев А.В., Мееров И.Б., Свистунов А.Н., Курылев А.Л., Сенин А.В., Шишков А.В., Корняков К.В., Сиднев А.А. «Параллельное программирование в системах с общей памятью. Инструментальная поддержка». Учебно-методические материалы по программе повышения квалификации «Технологии высокопроизводительных вычислений для обеспечения учебного процесса и научных исследований». Нижний Новгород, 2007, 110 с. 
\bibitem{Fox algorithm} Lab Course II Parallel Computing [Электронный ресурс] // URL: http://csis.uni-wuppertal.de/courses/scripts/\verb|skript_lab2.pdf|

\end{thebibliography}

\newpage

\section*{Приложение}
\addcontentsline{toc}{section}{Приложение}

\lstinputlisting[language=C++, caption=Sequential header]{../../../../modules/task_1/isaev_matrix_mult/matrix_mult.h}
\lstinputlisting[language=C++, caption=Sequential source]{../../../../modules/task_1/isaev_matrix_mult/matrix_mult.cpp}

\lstinputlisting[language=C++, caption=OpenMP header]{../../../../modules/task_2/isaev_fox_alg/fox_alg.h}
\lstinputlisting[language=C++, caption=OpenMP source]{../../../../modules/task_2/isaev_fox_alg/fox_alg.cpp}

\lstinputlisting[language=C++, caption=TBB header]{../../../../modules/task_3/isaev_fox_alg/fox_alg.h}
\lstinputlisting[language=C++, caption=TBB source]{../../../../modules/task_3/isaev_fox_alg/fox_alg.cpp}

\lstinputlisting[language=C++, caption=Threads C++11 header]{../../../../modules/task_4/isaev_fox_alg/fox_alg.h}
\lstinputlisting[language=C++, caption=Threads C++11 source]{../../../../modules/task_4/isaev_fox_alg/fox_alg.cpp}

\end{document}