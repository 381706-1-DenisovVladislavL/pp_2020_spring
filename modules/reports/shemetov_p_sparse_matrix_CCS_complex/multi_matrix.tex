\documentclass[14pt,a4paper,report]{ncc}

\usepackage{cmap}
\usepackage[T2A]{fontenc}
\usepackage[utf8]{luainputenc}
\usepackage[english, russian]{babel}
\usepackage[pdftex]{hyperref}
\pagestyle{plain}
\usepackage[a4paper, mag=1000, left=2.5cm, right=1cm, top=2cm, bottom=2cm, headsep=0.7cm, footskip=1cm]{geometry}
\usepackage{listings}
\usepackage{indentfirst}

\setlength{\parskip}{0.2cm}

\lstset{language=C++,
	basicstyle=\ttfamily,
	keywordstyle=\color{blue}\ttfamily,
	stringstyle=\color{red}\ttfamily,
	commentstyle=\color{green}\ttfamily,
	morecomment=[l][\color{magenta}]{\#}
}



\begin{document}
	\begin{titlepage}
		
		\begin{center}
			Министерство науки и высшего образования Российской Федерации
		\end{center}
		
		\begin{center}
			Федеральное государственное автономное образовательное учреждение высшего образования \\
			Национальный исследовательский Нижегородский государственный университет им. Н.И. Лобачевского
		\end{center}
		
		\begin{center}
			Институт информационных технологий, математики и механики
		\end{center}
		
		\vspace{4em}
		
		\begin{center}
			\textbf{\LargeОтчет по лабораторной работе} \\
		\end{center}
		\begin{center}
			\textbf{\Large«Умножение разреженных матриц. Элементы комплексного типа. Формат хранения матрицы – столбцовый (CCS)»} \\
		\end{center}
		
		\vspace{11em}
		\hfill\parbox{6.5cm}{
			\textbf{Выполнил:} \\ студент группы 381708-1 \\ Шеметов Ф.А.\\
			\\
			\textbf{Проверил:}\\ доцент кафедры МОСТ, \\ кандидат технических наук \\ Сысоев А. В.
		}
		
		\vspace{\fill}
		\begin{center} Нижний Новгород \\ 2020 \end{center}
		
	\end{titlepage}
	
	\setcounter{page}{2}
	
	\tableofcontents
	\newpage
	
	\section*{Введение}
	\addcontentsline{toc}{section}{Введение}
	
	Разрежённой матрицей называются матрицы, которые преимущественно состоят из нулевых элементов. Возникают разрежённые матрицы при решение различных задач связанных с научной и инженерной областью, где присутствует большое число неизвестных, связанных между собой уравнениями.
	\par Так как в разрежённых матрицах содержится большинство значений в виде нулей, то любая арифметическая операция с этими матрицами увеличивает лишние вычислительные затраты. Для того чтобы снизить вычислительные затраты, были придуманы разные способы хранений разрежённых матриц и один из них --- Формат хранения CCS\footnote{Compressed Column Storage}. 
	\par Формат хранения CCS представляет собой структуру данных, которая позволяет хранить матрицу в виде трех массивов. Первый массив хранит ненулевые значения элементов матрицы. Второй массив хранит номера строк для каждого элемента. Третий массив хранит индекс начала каждого столбца.
	\par Сам формат хранения CCS предоставляет минимальные требования к памяти и значительно уменьшает вычислительные затраты при арифметических операциях с разрежёнными матрицами.
	\par Цель лабораторной работы --- изучение принципа хранения и разработка алгоритма умножения разрежённых матриц в формате CCS с использованием технологий параллельного программирования.
	\newpage
	
	\section*{Постановка задачи}
	\addcontentsline{toc}{section}{Постановка задачи}
	В данной лабораторной работе ставится задача в виде разработки нескольких проектов, где нужно реализовать алгоритм умножения разрежённых матриц в формате хранения CCS с элементами комплексного типа.
	\par Проект будет включать в себя:
	\begin{itemize}
		\item Набор юнит-тестов использующие Google C++ Testing Framework.
		\item Исходный и заголовочный файл. В заголовочном файле реализован интерфейс класса, а в самом исходном файле реализован код последовательного алгоритма умножения разрежённых матриц в формате хранения CCS или параллельный алгоритм с технологиями OpenMP, TBB, std::threads.
		\item файл CMake для сборки проекта.
	\end{itemize}
	\newpage
	
	
	\section*{Описание алгоритма}
	\addcontentsline{toc}{section}{Постановка задачи}
	
	
	
	
\end{document}