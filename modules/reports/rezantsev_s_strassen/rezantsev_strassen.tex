\documentclass{article}

\usepackage[T2A]{fontenc}
\usepackage[utf8]{luainputenc}
\usepackage[english, russian]{babel}
\usepackage[pdftex]{hyperref}
\usepackage[14pt]{extsizes}
\usepackage{listings}
\usepackage{color}
\usepackage{geometry}
\usepackage{enumitem}
\usepackage{multirow}
\usepackage{graphicx}
\usepackage{indentfirst}
\usepackage[ruled,vlined,linesnumbered]{algorithm2e}

\newenvironment{myalgorithm}[1][htb]
  {\renewcommand{\algorithmcfname}{Алгоритм}
   \begin{algorithm}[#1]%
  }{\end{algorithm}}

\setlength{\parskip}{3mm}
\geometry{a4paper,top=20mm,bottom=20mm,left=25mm,right=15mm}
\setlist{nolistsep, itemsep=0.5cm,parsep=0pt}

\usepackage{xcolor}

\colorlet{mygray}{black!30}
\colorlet{mygreen}{green!60!blue}
\colorlet{mymauve}{red!60!blue}
\colorlet{myred}{red!60!blue}

\lstset{
  backgroundcolor=\color{gray!10},  
  basicstyle=\fontsize{13}{13}\ttfamily,
  columns=fullflexible,
  breakatwhitespace=false,      
  breaklines=true,                
  captionpos=b,                    
  commentstyle=\color{mygreen}, 
  extendedchars=true,              
  frame=single,                   
  keepspaces=true,             
  keywordstyle=\color{blue},      
  language=c++,                 
  numbers=none,                
  numbersep=1pt,                   
  numberstyle=\tiny\color{blue}, 
  rulecolor=\color{mygray},        
  showspaces=false,               
  showtabs=false,                                  
  stringstyle=\color{mymauve},                         
  title=\lstname                
}

\makeatletter
\renewcommand\@biblabel[1]{#1.\hfil}
\makeatother

\begin{document}

\begin{titlepage}

\begin{center}
Министерство науки и высшего образования Российской Федерации \\
\vspace{5mm}
Федеральное государственное автономное образовательное учреждение высшего образования \\
Национальный исследовательский Нижегородский государственный университет им. Н.И. Лобачевского \\
\vspace{1cm}
Институт информационных технологий, математики и механики \\
\vspace{5cm}
\textbf{\large Отчет по лабораторной работе} \\
\vspace{8mm}
\textbf{\Large «Алгоритм Штрассена»} \\
\end{center}

\vspace{3cm}

\newbox{\lbox}
\savebox{\lbox}{\hbox{text}}
\newlength{\maxl}
\setlength{\maxl}{\wd\lbox}
\hfill\parbox{7cm}{
\hspace*{5cm}\hspace*{-5cm}\textbf{Выполнил:} \\ студент группы 381706-1 \\ Резанцев С. А.\\
\\
\hspace*{5cm}\hspace*{-5cm}\textbf{Проверил:}\\ доцент кафедры МОСТ, \\ кандидат технических наук \\ Сысоев А. В.
}

\vspace{\fill}

\begin{center} 
Нижний Новгород \\ 2020
\end{center}
\end{titlepage}

\setcounter{page}{2}

\tableofcontents

\newpage

\section{Введение}
Алгоритм Штрассена был разработан Фолькером Штрассеном в 1969 году и предназначен для быстрого умножения матриц. Несмотря на то, что алгоритм Штрассена является не самым быстрым из существующих алгоритмов умножения матриц, он довольно прост в реализации. Алгоритм Штрассена имеет меньшую асимптотическую сложность чем традиционный алгоритм умножения и потенциально быстрее на больших размерах матриц. 

\par В настоящее время множество задач требуют проведение трудоемких операций над большими плотными матрицами за минимальное время.  Для решения этой проблемы предлагается ускорить алгоритм Штрассена с помощю распараллеливания вычислений. Однако рекурсивная природа данного алгоритма представляет определенную сложность для эффективного распараллеливания на современных вычислительных системах и использования данных из памяти.

\par Для разработки параллельных программ широко шспользуется применение тех или иных библиотек, обеспечивающих определенный программный интерфейс (API). Наиболее распространенными API являются: OpenMP, TBB и std::thread. 

\par Целью данной лабораторной работы является разработка параллельного алгоритма Штрассена и нахождение наиболее подходящего программного интерфейса для распараллеливания данной задачи.

\newpage

\section{Постановка задачи}
В рамках данной лабораторной работы нужно реализовать параллельный алгоритм решения задачи глобальной оптимизации.

\par Необходимо реализовать следующие компоненты:
 
\begin{enumerate}
\item Последовательный алгоритм Штрассена.
\item Параллельный алгоритм Штрассена с использованием технологий OpenMP, TBB и std::threads.
\item Вспомогательные функции для проверки работоспособности и эффективности
\end{enumerate}

\par Программное решение будет выглядеть следующим образом: 
 
\begin{enumerate}
\item Модули, содержащие реализацию библиотек. 
\item Наборы автоматических тестов с использованием Google C++ Testing Framework.
\end{enumerate}


\newpage

\section{Метод решения}
Метод Штрассена для умножения матриц размера $2^n$ основан на рекурсивном делении каждой перемножаемой матрицы на 4 подматрицы и выполнении операций над ними.  Если размер умножаемых матриц не является натуральной степенью двойки, мы дополняем исходные матрицы дополнительными нулевыми строками и столбцами. При этом мы получаем удобные для рекурсивного умножения размеры, но теряем в эффективности за счёт дополнительных ненужных умножений.

\par Пусть \verb|A, B| ~--- две квадратные матрицы размера $2^n$. Матрица \verb|C| получается по формуле:

\begin{center} 
 $C=AB$
\end{center}

\par Тогда подматрицы матрицы \verb|C| можно описать следующими формулами:

\begin{center} 
$C_{11}=P_1+P_4-P_5+P_7$

$C_{12}=P_3+P_5$

$C_{21}=P_2+P_4$

$C_{22}=P_1-P_2+P_3+P_6,$
\end{center}

где

\begin{center} 
$P_1=(A_{11}+A_{22})*(B_{11}+B_{22})$

$P_2=(A_{21}+A_{22})*B_{11}$

$P_3=A_{11}*(B_{12}-B_{22})$

$P_4=A_{22}*(B_{21}-B_{11})$

$P_5=(A_{11}+A_{12})*B_{22}$

$P_6=(A_{21}-A_{11})*(B_{11}+B_{12})$

$P_7=(A_{12}-A_{22})*(B_{21}+B_{22})$
\end{center}

\par Рекурсивный процесс продолжается n раз, до тех пор пока размер матриц  $C_{ij}$ не станет достаточно малым, далее используют обычный метод умножения матриц. Это делают из-за того, что алгоритм Штрассена теряет эффективность по сравнению с обычным на малых матрицах в силу большего числа сложений. Оптимальный размер матриц для перехода к обычному умножению зависит от характеристик процессора и языка программирования и на практике лежит в пределах от 32 до 128.

\newpage

\section{Схема распараллеливания}
Для распараллеливания данного алгоритма вполне достаточно выполнить параллельно семь операций: 

\begin{center} 
$P_1=(A_{11}+A_{22})*(B_{11}+B_{22})$

$P_2=(A_{21}+A_{22})*B_{11}$

$P_3=A_{11}*(B_{12}-B_{22})$

$P_4=A_{22}*(B_{21}-B_{11})$

$P_5=(A_{11}+A_{12})*B_{22}$

$P_6=(A_{21}-A_{11})*(B_{11}+B_{12})$

$P_7=(A_{12}-A_{22})*(B_{21}+B_{22})$
\end{center}

\par Кроме того, возможно параллельное выполнение последующих операций сложения, а также разделения и слияния подматриц, однако распараллеливание данных операций не дает значительного ускорения.

\newpage

\section{Описание программной реализации}
Для реалицации параллельной версии алгоритма Штрассена будут использоваться: OpenMP, TBB и std::threads. Стоит уточнить, что схема распараллеливания для каждой из технологий одна и таже, менятся будут лишь методы.

\subsection{OpenMP}
Для параллелизации используется директива 
\verb|#pragma omp parallel sections|. 
\par Реализация алгоритма Штрассена с помощью OpenMP:

\vspace{10pt}
\begin{lstlisting}
#pragma omp parallel sections default(shared)
  {
#pragma omp section
    {
      std::vector<double> t1(h * h);
      std::vector<double> t2(h * h);
      sumMatrix(a11, a22, &t1);
      sumMatrix(b11, b22, &t2);
      strassen_omp(t1, t2, &p1);
    }
#pragma omp section
    {
      std::vector<double> t1(h * h);
      sumMatrix(a21, a22, &t1);
      strassen_omp(t1, b11, &p2);
    }
#pragma omp section
    {
      std::vector<double> t1(h * h);
      subtMatrix(b12, b22, &t1);
      strassen_omp(a11, t1, &p3);
    }
#pragma omp section
    {
      std::vector<double> t1(h * h);
      subtMatrix(b21, b11, &t1);
      strassen_omp(a22, t1, &p4);
    }
#pragma omp section
    {
      std::vector<double> t1(h * h);
      sumMatrix(a11, a12, &t1);
      strassen_omp(t1, b22, &p5);
    }
#pragma omp section
    {
      std::vector<double> t1(h * h);
      std::vector<double> t2(h * h);
      subtMatrix(a21, a11, &t1);
      sumMatrix(b11, b12, &t2);
      strassen_omp(t1, t2, &p6);
    }
#pragma omp section
    {
      std::vector<double> t1(h * h);
      std::vector<double> t2(h * h);
      subtMatrix(a12, a22, &t1);
      sumMatrix(b21, b22, &t2);
      strassen_omp(t1, t2, &p7);
    }
  }
\end{lstlisting}
\vspace{-25pt}

\subsection{TBB}
Для параллелизации используется \verb|tbb::task_group|.

\par Реализация алгоритма Штрассена с помощью TBB:

\vspace{10pt}
\begin{lstlisting}
  tbb::task_scheduler_init init(8);
  tbb::task_group tasks;
  tasks.run(
      [=] { strassen(sumMatrix(a11, a22, n), sumMatrix(b11, b22, n), p1); });
  tasks.run([=] { strassen(sumMatrix(a21, a22, n), b11, p2); });

  tasks.run([=] { strassen(a11, subtMatrix(b12, b22, n), p3); });
  tasks.run([=] { strassen(a22, subtMatrix(b21, b11, n), p4); });

  tasks.run([=] { strassen(sumMatrix(a11, a12, n), b22, p5); });
  tasks.run(
      [=] { strassen(subtMatrix(a21, a11, n), sumMatrix(b11, b12, n), p6); });
  tasks.run(
      [=] { strassen(subtMatrix(a12, a22, n), sumMatrix(b21, b22, n), p7); });
  tasks.wait();
\end{lstlisting}
\vspace{-25pt}

\subsection{std::threads}
Для параллелизации используются объекты \verb|std::thread|.

\par  Реализация алгоритма Штрассена с помощью std::threads:

\vspace{10pt}
\begin{lstlisting}
  std::thread tread1(strassen, sumMatrix(a11, a22, n),
                     sumMatrix(b11, b22, n), &p1);
  std::thread tread2(strassen, sumMatrix(a21, a22, n), b11, &p2);
  
  std::thread tread3(strassen, a11, subtMatrix(b12, b22, n), &p3);
  std::thread tread4(strassen, a22, subtMatrix(b21, b11, n), &p4);
  
  std::thread tread5(strassen, sumMatrix(a11, a12, n), b22, &p5);
  std::thread tread6(strassen, subtMatrix(a21, a11, n),
                     sumMatrix(b11, b12, n), &p6);
  std::thread tread7(strassen, subtMatrix(a12, a22, n),
                     sumMatrix(b21, b22, n), &p7);
  tread1.join();
  tread2.join();
  tread3.join();
  tread4.join();
  tread5.join();
  tread6.join();
  tread7.join();
\end{lstlisting}
\vspace{-25pt}

\newpage

\section{Подтверждение корректности}
Для подтверждения корректности в программе представлен набор тестов, разработанных с помощью использования Google C++ Testing Framework.


\par Тесты проверяют корректную работу последовательного и параллельного алгоритмов Штрассена и сравнивают с результатами традиционного алгоритма умножения матриц.


\newpage

\section{Результаты экспериментов}
Вычислительные эксперименты проводились на ПК со следующей аппаратной конфигурацией:

\begin{itemize}
\item Операционная система: Windows 10 Home, версия 1903, сборка 18362.535
\item Процессор: Intel Core i5-8250U 1.6GHz
\item Число ядер: 4;
\item Оперативная память: 8 ГБ
\item Версия Visual Studio: 2017
\end{itemize}

\par В таблице 1 приведены результаты экспериментов. 

\par Размер матрицы: 1024x1024.

\begin{table}[!h]
\caption{Ускорение алгоритма в зависимости от числа потоков.}
\centering
\begin{tabular}{|c|c|c|c|}
\hline
	{\begin{tabular}[c]{@{}c@{}}Количество\\ потоков\end{tabular}} & 
	{\begin{tabular}[c]{@{}c@{}}Ускорение\\OpenMP\end{tabular}} & 
	{\begin{tabular}[c]{@{}c@{}}Ускорение\\TBB\end{tabular}} & 
	{\begin{tabular}[c]{@{}c@{}}Ускорение\\std\end{tabular}}
\\ \cline{1-4}

1	& 0.997	 & 1.007	& 0.980	\\ \hline
2   & 1.678     & 1.642    	& 1.599     \\ \hline
4   & 2.514     & 2.444    	& 2.386     \\ \hline
\end{tabular}
\end{table}

\par Следовательно, для распараллеливания алгоритма Штрассена лучшей является технология OpenMP.

\par Следует отметить, что распараллеливане слияния и разделения матриц, а также операций сложения, не дают существенных преимуществ: изменение времени выполнения программы на уровне погрешностей. Проверено с использованием технологии OpenMP.

\newpage

\section{Заключение}
В результате лабораторной работы были разработаны библиотеки, реализующие алгоритм Штрассена.

\par Реализация параллельной версии алгоритма с использованием технологий OpenMP, TBB и std::threads была успешно достигнута, о чем говорят результаты экспериментов, проведенных в ходе работы. Они показывают, что лучшей технологией для реализации параллельного алгоритма Штрассена является OpenMP.

\par Для подтверждения корректности работы программы были разработаны и доведены до успешного выполнения тесты, созданные с использованием Google C++ Testing Framework.

\newpage

\section{Список литературы}
\begin{enumerate}
\bibitem{OMP} Гергель В.П. Учебный курс «Введение в методы параллельного программирования», раздел «Параллельное программирование с использованием OpenMP». Нижний Новгород, ННГУ, 2007, 33 с.

\bibitem{TBB} А.А. Сиднев, А.В. Сысоев, И.Б. Мееров. Учебный курс «Технологии разработки параллельных программ», раздел «Создание параллельной программы», «Библиотека Intel Threading Building Blocks~--- краткое описание». Нижний Новгород, ННГУ, 2007, 29 с. 

\bibitem{STD} Справочник по конструкциям языка C++, раздел std::thread // URL: https://ru.cppreference.com/

\end{enumerate}

\newpage

\section{Приложение}

\subsection{Последовательный алгоритм Штрассена}

\lstinputlisting[language=C++]{../../../../modules/task_1/rezantsev_s_strassen/strassen.h}
\lstinputlisting[language=C++]{../../../../modules/task_1/rezantsev_s_strassen/strassen.cpp}
\lstinputlisting[language=C++]{../../../../modules/task_1/rezantsev_s_strassen/main.cpp}

\subsection{OpenMP}

\lstinputlisting[language=C++]{../../../../modules/task_2/rezantsev_s_strassen_omp/strassen.h}
\lstinputlisting[language=C++]{../../../../modules/task_2/rezantsev_s_strassen_omp/strassen.cpp}
\lstinputlisting[language=C++]{../../../../modules/task_2/rezantsev_s_strassen_omp/main.cpp}

\subsection{TBB}

\lstinputlisting[language=C++]{../../../../modules/task_3/rezantsev_s_strassen_tbb/strassen.h}
\lstinputlisting[language=C++]{../../../../modules/task_3/rezantsev_s_strassen_tbb/strassen.cpp}
\lstinputlisting[language=C++]{../../../../modules/task_3/rezantsev_s_strassen_tbb/main.cpp}

\subsection{std::threads}

\lstinputlisting[language=C++]{../../../../modules/task_4/rezantsev_s_strassen_std/strassen.h}
\lstinputlisting[language=C++]{../../../../modules/task_4/rezantsev_s_strassen_std/strassen.cpp}
\lstinputlisting[language=C++]{../../../../modules/task_4/rezantsev_s_strassen_std/main.cpp}


\end{document}